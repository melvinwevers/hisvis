\documentclass[12pt]{article}

\title{HiSVis}
\author{Melvin Wevers}
\date{\today}
 
\begin{document}
 
\maketitle
 
\section{Introduction}

Picture collection of press agency \textit{De Boer} for the period 1945-2004. The picture collection focuses on national and regional events, although over time the focus gradually shifted to the region Kennemerland. This region just north of Amsterdam includes cities such as Haarlem, Zandvoort, Bloemendaal, and Alkmaar.
%
In 1962, founder Cees de Boer was awarded the World Press Photo and the Silver Camera (\textit{Zilveren Camera}). Cees' son Joppe won the national photo journalist award in 1969. The press agency provided pictures to the Regional newspaper Haarlems Dagblad. The photo collection depicted a wide range of events, ranging from the first and only show of The Beatles in Blokker to opening of restaurants, and sport events.

The photo collection consists of approximately 2 million negatives and metadata. The metadata is based on topic cards and logs kept by the photo agency. The inventory contains data for the period 1945-1990. The agency detailed what was depicted using topic cards. Moreover, for the period 1952-2004, De Boer kept logs. For the period 1952-1990, these have been transcribed using volunteers, the remainder is in the process of transcription with a deadline in juli 2018.  The negatives of the pictures are currently being digitised, and these will be linked to the metadata using identifiers. These pictures, however, are not segmented.

In this paper, we explore how well existing scene detection algorithm can be fine-tuned for a historical photo collection. We will reflect on the creation of the training data, the training process, as well as the evaluation. Finally, we discuss the added benefits of this type of computer vision algorithm for heritage institutes and archives. 

\section{Data}

For this study, we relied on a subset of data that had already been digitized by the \textit{Noord-Hollands Archief}. This subset consisted of 2,545 images from their online repository, images digitized for the pilot, and the small negatives. 

We took the categorization scheme for Places-365 as a starting point. We augmented this scheme with information from the catalogue cards that the Boer used. We could not use these directly, because the often contain too explicit information, too specific category or category that were not visually represented. Together with archivists and cultural historians, we further optimized the categorization. This resulted in a training scheme of xxx categories. 
%TODO add number of categories 

We created a training set from the aforementioned data sets by hand-coding the images into the list of categories. For some categories we could not find an sufficient amount of training data. The bare minimum required was 15 images. These categories are: 
%TODO list categories without enough images

During the second phase of the project we will digitize the entire photo collection and for a subset of these images, we will use groups of annotators to provide them with annotations.

For this pilot study, we proceeding with the categories that had at least 15 images. 


Training, validation, and testing. Since the number of images in each category was distributed evenly, I decided to oversample. With a minimum of five images in the test set and 5 in the validation set. 




\section{Methods}
Scene detection .... 

We use a 


\section{Results}

\section{Discussion}


\end{document}